\documentclass[12pt]{report}
\usepackage{structure}
\begin{document}
\title{QFT: Some Notes on Concepts}
\author{Sam Crawford}
\maketitle
\chapter{Introduction}
This follows a slightly different path to the Part III course. Currently I intend to assume a solid understanding of everything up until the interaction picture. The main intention of these notes is really to provide solid justification for the Feynman rules for $\phi^4$ theory, (pseudo-)scalar Yukawa theory, and QED.


\chapter{Interaction Picture}

\section{Set Up: $\phi^4$ Theory}

\paragraph{}In essence, the aim here is to find the field-theoretic form of perturbation theory. The idea is to begin with a `free' Lagrangian density (loosely defined as a form that is quadratic in a field and its derivatives) and add higher order terms, presumed to be sufficiently small. The simplest example of this is the aforementioned $\phi^4$ theory. One might as why the first logical step is not $\phi^3$, and [I SHOULD REALLY ATTEMPT TO ANSWER THAT]. So, our first interacting theory can be written down as \footnote{The $4!$ term is ultimately chosen for convenience, as will be shown later.}

	\begin{equation}
		\mathcal{L} = \underbrace{\tfrac{1}{2} \left( \partial_\mu \phi \right) \left( \partial^\mu \phi \right) - \tfrac{1}{2} m^2 \phi^2}_{\text{Free field}} - \underbrace{\tfrac{\lambda}{4!} \phi^4}_{=: \mathcal{L}_{\text{int}}}.
	\end{equation}

In fact, if we stay with this sign convention, by noting that $\mathcal{L}_{\text{int}}$ does not depend on any derivatives of $\phi$, we see that the interaction does not affect the definition of the canonically conjugate momenta. Importantly, this means that the Hamiltonian for this theory is simply

	\begin{equation}
		H = \int_{\mathbb{R}^3} \mathrm{d}\mathbf{x} \left( \mathcal{H}_\text{KG} + \mathcal{L}_\text{int} \right).
	\end{equation}
	
Here, $\mathcal{H}_\text{KG}$ refers to the Klein-Gordon Hamiltonian we have previously encountered. Due to this expression, we shall usually refer to the interaction Hamiltonian ${H_\text{int} := \int \mathrm{d}\mathbf{x} \, \mathcal{L}_\text{int}}$.

\paragraph{}This is a good position to be in, as our new Hamiltonian is the sum of a Hamiltonian whose solutions we know explicitly, and a term that we assume to be `sufficiently small' for whatever purposes we require.

\section{Pictures: Heisenberg vs Schr\"odinger}

\paragraph{}When constructing free quantum field theories, it is most convenient to adopt the Schr\"odinger picture, which can be summarised as follows:

	\begin{itemize}
		\item Operators do not evolve over time, thus fields depend only on spatial variables, $\partial_0 \phi = 0$.
		\item Time evolution is instead accounted for in quantum states, evolving according to the Schr\"odinger equation $\hat{H} |\Psi \rangle = i\hbar \partial_0 |\Psi \rangle$.\footnote{Confusing as it is, recall that $\Psi$ here is a label for a \textit{state}, it is \textit{not} a field.}
	\end{itemize}
	
In this picture, we can easily Fourier transform our operators so that they depend on momenta instead of positions. The drawback is that we technically cannot even formulate a Hamiltonian, as the canonically conjugate momenta depend on the time derivatives of the field in the Lagrangian. The solution is to just leave them in the Lagrangian and ignore the fact that they vanish.

\paragraph{} Alternatively, one could use the Heisenberg picture, which assumes the following:

	\begin{itemize}
		\item States do not evolve over time, thus $\partial_0 |\Psi\rangle = 0$.
		\item Time evolution of operators is governed by \textit{Heisenberg's equation}
			\begin{equation}
				i \partial_0 \hat{\mathcal{O}} = [\hat{\mathcal{O}},\hat{H}]
			\end{equation}
	\end{itemize}
	
%The reason this is useful is quite subtle (I think). The \textit{space of basis eigenstates}, not to be confused with the eigenspaces, is itself a 3-dimensional vector space. Put another way, if we label our eigenstates as $|\mathbf{p}\rangle$, the space of appropriate choices for $\mathbf{p}$ is 3-D, even though the actual Hilbert space, $\mathcal{V} = \text{span}\{|\mathbf{p}\rangle\}_{\mathbf{p}\in\mathbb{R}^3}$, is infinite-dimensional. In the case of the Klein-Gordon field, this arises from the equation of motion
% PROBLEM: ACTUAL HILBERT SPACE IS FOCK SPACE. Which implies that single particle states aren't complete...


\chapter{Deriving Feynman Rules}

\paragraph{} Because nobody likes integrals, especially when the integration space exceeds 30 dimensions... we shall see how the computation can be simplified dramatically, and even represented diagramatically. The idea for this chapter is to introduce a bunch of machinery, first in real, scalar $\phi^4$ theory, then in as general a context as possible. For explicit examples of the rules, see [OTHER PDF]

\section{Normal Ordering}

[Remember to navigate around $\phi$ vs $\phi_I$ properly... good luck!]

\section{Wick's Theorem}

[Again, good luck sucker]

\section{Contractions}

\paragraph{} In [OTHER PDF] we defined the Wick contraction of two operators, or an operator and a state. We can now apply Wick's theorem to drastically simplify $S$-matrix calculations. Let's begin by enumerating the contractions for $\phi^4$ theory:

	\begin{align}
	\overbrace{\phi(x)\phi(y)} = \Delta_F(x-y); &&
	 \phi\overbrace{(x) |\mathbf{p}}\rangle	 = \int \frac{\mathrm{d}^3\mathbf{k}}{(2 \pi)^3} e^{i \mathbf{k}\cdot \mathbf{x}} |\mathbf{p},\mathbf{k}\rangle.
	\end{align}

Fortunately, as $\phi^4$ is a bosonic theory, $|\mathbf{p},\mathbf{k}\rangle = |\mathbf{k},\mathbf{p}\rangle$. So any mistakes in ordering the particles can be forgiven. We will have to be more careful when studying fermionic theories, however.

\paragraph{} A final piece of machinery before we can begin computing is the overlap between multi-particle states. It is possibly easiest to simply write down the explicit case for 2-particles, then justify it and its generalisation.

	\begin{align}
		\langle \mathbf{p}' , \mathbf{q}' | \mathbf{p} , \mathbf{q} \rangle = (2\pi)^6 \left[ \delta( \mathbf{p}' -  \mathbf{p})\delta( \mathbf{q}' -  \mathbf{q}) + \delta( \mathbf{p}' -  \mathbf{q})\delta(  \mathbf{q}' -  \mathbf{p} ) \right].
	\end{align}

fuck... energy shit

\end{document}