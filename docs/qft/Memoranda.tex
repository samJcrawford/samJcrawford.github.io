\documentclass[12pt]{report}
\usepackage{structure}
\usepackage{tikz-feynman}

\begin{document}
\title{QFT: Facts, Formulae and Functions}
\author{Sam Crawford}
\maketitle
\stepcounter{chapter}
\chapter{Propagators}

\section{Free Fields}

\paragraph{} Beginning with a free theory (e.g. Klein-Gordon), the physical interpretation of $\phi(x)|0\rangle$ is a state with one particle of the field $\phi$ existing at the event $x$. Intuitively, from the Heisenberg uncertainty principle, it's momentum is therefore maximally undetermined, thus there is a non zero probability of the particle being observed at $y$, so long as $x-y$ is time-like. Mathematically, this is represented by

	\begin{equation}	\label{PropagatorOverlap}
		\langle 0 | T\left\{ \phi(x) \phi(y) \right\} | 0 \rangle.
	\end{equation}

We make the canonical decision to time order the overlap as this then has the further interpretation that the particle is travelling from the earlier event to the later. This is the \textit{propagator}. For a scalar K-G theory, this can be written as

	\begin{equation}	\label{FeynmanAndContraction}
		\Delta_F(x-y) = \phi\overbrace{(x)\phi}(y) =
		\lim_{\epsilon \to 0} \int_\mathbb{M}\frac{\mathrm{d}^4p}{(2\pi)^4} \frac{i e^{-i \langle p , x-y \rangle}}{ \langle p, p \rangle - m^2 + i\epsilon}.
	\end{equation}
The two symbols newly defined here have slightly different meanings in different context, and it will be vital to understand how they differ for various field types (spinor, vector, etc...) as well as when interactions are included into the theory.

\paragraph{}The first is the \textit{Feynman propagator}. This is the object most closely related to the expression \eqref{PropagatorOverlap}, as it is defined as the time ordered ordered overlap of the states $\phi(x) |0 \rangle$ and $\phi(y) | 0 \rangle$. This will stray from \eqref{PropagatorOverlap} if the field is not self-adjoint (as is the case with a real scalar field).

\section{Contractions}

\paragraph{} The second expression in \eqref{FeynmanAndContraction} is the \textit{contraction} of a pair of operators. It is defined in terms of the \textit{normal ordering} of the fields. At it's most general, the contraction of a pair of operators is defined as

	\begin{equation}
		\overbrace{AB} := T\{AB\} - :AB:
	\end{equation}

Naturally this definition is dependent on the definition of normal ordering, which is reviewed [HERE]. Furthermore, we can define \textbf{contraction of an operator with a state}, as we are using a Fock space, which has a basis generated by $a^\dagger_\mathbf{p}|0\rangle, a^\dagger_\mathbf{p}a^\dagger_\mathbf{q}|0\rangle,\ldots$, with similar terms added for each new field we consider. For a momentum eigenstate of the free theory, this contraction is defined as

	\begin{equation}
		\overbrace{A|\mathbf{p}_1,\ldots,\mathbf{p}_i} ,\ldots \mathbf{p}_n \rangle
		= \overbrace{Aa^\dagger_{\mathbf{p}_i}} |\mathbf{p}_1,\ldots,\mathbf{p}_{i-1},\mathbf{p}_{i+1} ,\ldots \mathbf{p}_n \rangle
	\end{equation}

\chapter{Feynman Rules}

\begin{figure}
\centering
	\feynmandiagram[horizontal=f2 to f3] {
		f1 -- [fermion] f2 -- [fermion] f3 -- [fermion] f4,
		f2 -- [photon] p1,
		f3 -- [photon] p2,
	};
\end{figure}



\end{document}