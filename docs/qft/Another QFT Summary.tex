\documentclass[12pt]{article}

\usepackage{../noteStructure}


\begin{document}
	
	\title{QFT: A Brief Summary}
	\author{Sam Crawford}
	
	\maketitle

\section{Classical Field Theory}
	
\begin{noteEquation}[Euler-Lagrange Equations]
	\begin{equation}
		\partial_\mu \left( \frac{\partial \mathcal{L}}{\partial (\partial_\mu \phi_a)} \right) - \frac{\partial \mathcal{L}}{\partial \phi_a} = 0.
	\end{equation}
\end{noteEquation}

\begin{noteEquation}[Noether's Theorem]
An infinitesimal transformation $\delta \phi_a$ is a symmetry of a field theory if $\delta \mathcal{L} = \partial_\mu \chi^\mu$, then
	\begin{equation}
		\partial_\mu \left( \frac{\partial \mathcal{L}}{\partial (\partial_\mu \phi_a)} \delta \phi_a - \chi^\mu \right) = 0.
	\end{equation}
\end{noteEquation}

\begin{noteEquation}[Stress-Energy Tensor]
	A special case of the above. If the variation is obtained by $x^\mu \to x^\mu + \epsilon^\mu$, then
		\begin{equation}
			\partial_\mu \left( \frac{\partial \mathcal{L}}{\partial (\partial_\mu \phi_a)} \epsilon^\nu \partial_\nu \phi_a - \epsilon^\mu \mathcal{L} \right) = \epsilon^\nu \partial_\mu {T^\mu}_\nu = 0.
		\end{equation}
\end{noteEquation}

asdfasdf
\blockquote{Hi}

\end{document}