\documentclass[12pt]{article}

\usepackage{../noteStructure}

\title{Quantum Field Theory\\
\Large{Summary Revision Notes}}
\author{Sam Crawford}
\date{\today}

\begin{document}

\maketitle

\tableofcontents
\pagebreak

\section{Classical Field Theory}

\begin{definition}[Lagrangian density and action]
	The \textbf{Lagrangian density} of a field theory, which can be thought of \textit{as} the field theory itself is a function
		\begin{align}
			\mathcal{L} = \mathcal{L}(\phi_a, \partial_\mu \phi_a).
		\end{align}
	This is used mainly to define the \textbf{action} of the field theory
		\begin{align}
			S[\phi_a] = \int_\mathbb{M} \mathrm{d}^4 x \, \mathcal{L}(\phi_a, \partial_\mu \phi_a).
		\end{align}
\end{definition}

\begin{prop}[Euler-Lagrange equations]\label{prop:E-L_Equations}
	The action of a field theory with Lagrangian density $\mathcal{L}$ is minimised when the \textbf{Euler-Lagrange equations}
		\begin{align}
			\partial_\mu \left( \frac{\partial \mathcal{L}}{\partial (\partial_\mu \phi_a)} \right) - \frac{\partial \mathcal{L}}{\partial \phi_a} = 0
		\end{align}
	are satisfied.
\end{prop}
\begin{proof}
	See \cref{proof:E-L_Equations} for proof.
\end{proof}

\begin{example}[Klein-Gordon Field]
	The \textbf{Klein-Gordon} Lagrangian for a real scalar field $\phi$ is
		\begin{align}
			\mathcal{L} = \tfrac{1}{2} (\partial_\mu \phi) (\partial^\mu \phi) - \tfrac{1}{2} m^2 \phi^2.
		\end{align}
	The Euler-Lagrange equation for which, called the \textit{Klein-Gordon equation} is
		\begin{align}
			\partial^\mu \partial_\mu \phi + m^2 \phi = (\square + m^2) \phi = 0.
		\end{align}
\end{example}

\begin{example}[Electromagnetic Field]
	In a vacuum (i.e. no charged particles), \textbf{Maxwell's electromagnetism} is given by the Lagrangian
		\begin{align}
			\mathcal{L} = -\tfrac{1}{4} F_{\mu\nu} F^{\mu\nu}.
		\end{align}
	Variation of which produces Maxwell's equations
		\begin{align}
			\partial_\mu F^{\mu\nu} = 0.
		\end{align}
\end{example}

\begin{prop}[Noether's Theorem]\label{prop:Noether}
	If a field theory has an action which is invariant under the action of some Lie group, then there is an associated \textit{conserved current} $j^\mu$ such that
		\begin{align}
			\partial_\mu j^\mu = 0, && \Rightarrow && \frac{d}{dt} \left( \int_V \mathrm{d}^3 x \, j^0 \right) = \int_{\partial V} j^i \, \mathrm{d}S_i.
		\end{align}
\end{prop}
\begin{proof}
	See \cref{proof:Noether} for proof.
\end{proof}

\begin{example}[The Energy-Momentum Tensor]
	A common external symmetry in classical mechanics is that of spatial translation. Consider the action of an infinitesimal translation
		\begin{equation}
		x^\mu \to X^\mu + \epsilon^\mu \quad \Rightarrow \quad \phi_a \to \phi_a + \epsilon^\mu \partial_\mu \phi_a
		\end{equation}
\end{example}


\pagebreak
\appendix
\section{Proofs}
\numberwithin{equation}{subsection}

\subsection{Proof of Proposition \ref{prop:E-L_Equations}}\label{proof:E-L_Equations}
	\begin{proof}
		For now, we just treat $\phi_a$ and $\partial_\mu \phi_a$ as variables, not functions. As such the variation of the Lagrangian density is
			\begin{align}
				\delta \mathcal{L} = \frac{\partial \mathcal{L}}{\partial \phi_a} \delta \phi_a + \frac{\partial \mathcal{L}}{\partial ( \partial_\mu \phi_a )} \delta (\partial_\mu \phi_a ).
			\end{align}
		Next, for any reasonable variation of $\phi_a$, we would expect that $\delta	(\partial_\mu \phi_a) = \partial_\mu ( \delta \phi_a )$. Assuming this is the case, then integrating by parts (noting that $\mathbb{M}$ has no boundary), the variation of the action is
			\begin{align}
				\delta S = \int_\mathbb{M }\mathrm{d}^4 x \, \left[ \frac{\partial \mathcal{L}}{\partial \phi_a} - \partial_\mu \left( \frac{\partial \mathcal{L}}{\partial (\partial_\mu \phi)} \right) \right] \delta \phi_a .
			\end{align}
		Assuming this vanishes for any variation $\delta \phi_a$ means, by the fundamental lemma of the calculus of variations, that the term in square brackets must also vanish. This results in the Euler-Lagrange equations.
	\end{proof}
	
\subsection{Proof of Proposition \ref{prop:Noether} (Nother's Theorem)}\label{proof:Noether}
	\begin{proof}
		The action of (part of) a Lie group can be represented by action of the corresponding Lie algebra. Basically, we consider an `infinitesimal' transformation
			\begin{align}\label{NoetherVariation}
				\phi_a \to \phi_a + \delta \phi_a = \phi_a + X_a(\phi).
			\end{align}
		The condition for invariance, that $\int_\mathbb{M} \delta \mathcal{L} \, \mathrm{d}^4 x$ vanishes, allows us to say that $\delta L = \partial_\mu F^\mu$ (ignoring cohomology). Varying the Lagrangian using \eqref{NoetherVariation} gives us
			\begin{align}
				\delta \mathcal{L} 
				&= \frac{\partial \mathcal{L}}{\partial \phi_a} X_a + \frac{\partial \mathcal{L}}{\partial (\partial_\mu \phi_a )} \partial_\mu X_a \nonumber\\
				&= \left[ \frac{\partial \mathcal{L}}{\partial \phi_a} - \partial_\mu \left(\frac{\partial \mathcal{L}}{\partial (\partial_\mu \phi_a)} \right) \right] X_a + \partial_\mu \left( \frac{\partial \mathcal{L}}{\partial (\partial_\mu \phi_a)} X_a \right).
			\end{align}
	\end{proof}

\end{document}