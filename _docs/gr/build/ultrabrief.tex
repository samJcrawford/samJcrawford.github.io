%% 
%% This is file, `ultrabrief.tex',
%% generated with the extract package.
%% 
%% Generated on :  2018/05/30,17:30
%% From source  :  GR_Summary.tex
%% Using options:  active,generate=UltraBrief,,extract-cmd=section,extract-env={noteEquation}
%% 
\documentclass[12pt]{article}
\usepackage{../../../mainStructure}

\begin{document}

\section{Equivalence Principles}

\begin{noteEquation}[Newton's Law of Gravitation]
The differential form of Newtonian gravity is
\begin{equation}
\Delta \Phi = 4 \pi G \rho.
\end{equation}
The integral solution to this is
\begin{equation}
\phi(t,\mathbf{x}) = - G \int_{\mathbb{R}^3} \mathrm{d}^3y \, \frac{
\rho(t,\mathbf{y})
}{
| \mathbf{x} - \mathbf{y}|.
}
\end{equation}
\end{noteEquation}

\section{Manifolds and Tensors}

\section{The Metric Tensor}

\begin{noteEquation}[Geodesic Lagrangian]\label{geoLagrange}
To find geodesics on a Lorentzian manifold, we use a functional formula for the proper time, treating this as an action, the `Lagrangian' is
\begin{equation}
G(x(\lambda),\dot{x}(\lambda)) \coloneqq \sqrt{-g_{\mu\nu}(x) \dot{x}^\mu \dot{x}^\nu}.
\end{equation}
The proper time for a curve $x: [0,1] \hookrightarrow M$ is then
\begin{equation}
\tau[x] = \int_0^1 G(x(\lambda),\dot{x}(\lambda)) \mathrm{d}\lambda.
\end{equation}
\end{noteEquation}

\begin{noteEquation}[Geodesic Equation]
The Euler-Lagrange equations for \autoref{geoLagrange} reduce to the \textbf{geodesic equation}
\begin{equation}\label{eq:geodesic}
\frac{d^2 x^\mu }{d \tau^2} + \Gamma^\mu_{\nu\rho} \frac{d x^\nu}{d\tau} \frac{dx^\rho}{d\tau} = 0,
\end{equation}
where the \textbf{Christoffel symbols} are defined by
\begin{equation}\label{eq:Christoffel}
\Gamma^\mu_{\nu\rho} \coloneqq \tfrac{1}{2} g^{\mu\sigma} ( g_{\sigma \nu, \rho} + g_{\sigma \rho, \nu} - g_{\nu \rho, \sigma} ).
\end{equation}
\end{noteEquation}

\section{Covariant Derivative}

\begin{noteEquation}[Tensor Coordinate Transformation]
The generalisation of \autoref{prop:trans1} for an arbitrary $(r,s)$ tensor field is simply
\begin{equation}
T'^{\mu_1 \cdots \mu_r}_{\phantom{\mu_1 \cdots \mu_r} \nu_1 \cdots \nu_s} = \left( \frac{\partial x'^{\mu_1}}{\partial x^{\rho_1}} \right) \cdots \left( \frac{\partial x'^{\mu_r}}{\partial x^{\rho_r}} \right) \left( \frac{\partial x^{\sigma_1}}{\partial x'^{\nu_1}} \right) \cdots \left( \frac{\partial x^{\sigma_s}}{\partial x'^{\nu_s}} \right) T^{\rho_1 \cdots \rho_r}_{\phantom{\rho_1 \cdots \rho_r} \sigma_1 \cdots \sigma_s}
\end{equation}
\end{noteEquation}

\section{Physical Laws in Curved Spacetime}

\section{Curvature}

\begin{noteEquation}
By computing $R(e_\rho, e_\sigma) e_\nu$, we can obtain the basis-dependent form of the Riemann tensor
\begin{equation}
{R^\mu}_{\nu \rho \sigma} = \partial_\rho \Gamma^\mu_{\nu\sigma}
 - \partial_\sigma \Gamma^\mu_{\nu\rho} + \Gamma^\tau_{\nu\sigma}\Gamma^\mu_{\tau\rho} - \Gamma^\tau_{\nu\rho}\Gamma^\mu_{\tau\sigma}.
  \end{equation}
\end{noteEquation}

\begin{noteEquation}[Ricci Identity]
\begin{equation}
\nabla_c \nabla_d Z^a - \nabla_d \nabla_c Z^a = {R^a}_{bcd} Z^b.
\end{equation}
\end{noteEquation}

\begin{noteEquation}[Symmetries of the Riemann Tensor]
\begin{equation}
{R^a}_{b(cd)} = 0.
\end{equation}
For a torsion-free connexion:
\begin{equation}
{R^a}_{[bcd]} = 0.
\end{equation}
\textit{Bianchi identity:} (also for a torsion-free connexion)
\begin{equation}
{R^a}_{b[cd;e]} = 0.
\end{equation}
If $\nabla$ is the Levi-Civita connexion for some metric $g_{ab}$, then
\begin{equation}
R_{abcd} = R_{cdab}, \qquad R_{(ab)cd} = 0.
\end{equation}
\end{noteEquation}

\begin{noteEquation}[Geodesic Deviation]
Let $\nabla$ be a torsion-free connection, and let $T,S$ be vector fields such that $\nabla_T T = 0$, and $[T,S] = 0$. Then
\begin{equation}
\nabla_T \nabla_T S = R(T,S) T.
\end{equation}
\end{noteEquation}

\begin{noteEquation}[Contracted Bianchi Identity]
\begin{align}
\nabla^aG_{ab} = 0, \quad \Leftrightarrow \quad \nabla^a R_{ab} - \frac{1}{2} \nabla_b R = 0.
\end{align}
\end{noteEquation}

\begin{noteEquation}[Einstein Equation]
\begin{equation}
G_{ab} = 8\pi G T_{ab}.
\end{equation}
If a vacuum, this reduces to
\begin{equation}
R_{ab} = 0.
\end{equation}
If we wish to include the cosmological constant, we then have
\begin{equation}
G_{ab} + \Lambda g_{ab} = 8 \pi G T_{ab}.
\end{equation}
\end{noteEquation}

\section{Diffeomorphisms and the Lie Derivative}

\begin{noteEquation}[Coordinate Based Pullback/Pushforward]
Let $\phi: M \to N$ be a diffeomorphism, and let $(x^\mu)$ be a set of coordinates on $M$, and $(y^\mu)$ a set of coordinates on $N$. Then
\begin{align}
\left( \phi_* \frac{\partial}{\partial x^\mu} \right)
= \left( \frac{\partial y^\nu}{\partial x^ \mu} \right) \frac{\partial }{\partial y^\nu},
\qquad
\left( \phi^* dy^\mu \right) = \left( \frac{\partial y^\mu }{\partial x^\nu} \right) dx^\nu.
\end{align}
The inverse operations, allowed by the fact that $\phi$ is a diffeomorphism, are then
\begin{align}
\left( \phi^* \frac{\partial}{\partial y^\mu} \right)
= \left( \frac{\partial x^\nu}{\partial y^\mu} \right) \frac{\partial}{\partial x^\nu},
\qquad
\left( \phi_* dx^\mu \right)
= \left(\frac{\partial x^\mu }{\partial y^\nu} \right) dy^\nu.
\end{align}
Thus, for a tensor $S$ of type $(r,s)$ on $M$, and $T$ on $N$, we have
\begin{align}
{\left( \phi_* S \right)^{\mu_1, \cdots, \mu_r}}_{\nu_1, \cdots \nu_s} = \left( \frac{\partial y^{\mu_1}}{\partial x^{\alpha_1}}\right) \cdots \left( \frac{\partial y^{\mu_r}}{\partial x^{\alpha_r}}\right)
\left( \frac{\partial x^{\beta_1}}{\partial y^{\nu_1} }\right) \cdots
\left( \frac{\partial x^{\beta_s}}{\partial y^{\nu_s} }\right)
{S^{\alpha_1 \cdots \alpha_r}}_{\beta_1 \cdots \beta_s}.
\end{align}
\end{noteEquation}

\begin{noteEquation}[Killing's Equation]
A vector field $X$ is a \textbf{Killing vector field} if its flows are isometries of $g$, equivalently $\mathcal{L}_X g = 0$. This can be show to be equivalent to \textit{Killing's equation}
\begin{equation}
\nabla_a X_b + \nabla_b X_a = 0.
\end{equation}
Or, equivalently
\end{noteEquation}

\section{Linearised Theory}

\begin{noteEquation}[Linearly Perturbed Metric]
Assuming that space is `almost flat' we can find a chart such that
\begin{equation}
g_{\mu\nu} = \eta_{\mu\nu} + h_{\mu\nu},
\end{equation}
such that $h_{\mu\nu} \sim \mathcal{O}(\epsilon)$ and $\epsilon^2 \approx 0$. The inverse metric is then
\begin{equation}
g^{\mu \nu} = \eta^{\mu\nu} + h^{\mu\nu},
\end{equation}
where $h^{\mu\nu} = \eta^{\mu\rho} \eta^{\nu\sigma} h_{\rho \sigma} + \mathcal{O}(\epsilon^2)$.
\end{noteEquation}

\begin{noteEquation}[Linearised Levi-Civita]
Given the above linearisation, the Christoffel symbols of the Levi-Civita connexion can be written
\begin{equation}
\Gamma^\mu_{\nu\rho} = \frac{1}{2}
\eta^{\mu\sigma} (h_{\sigma \nu,\rho} + h_{\sigma \rho, \nu} - h_{\nu\rho, \sigma} ).
\end{equation}
This leads to the linearised Riemann tensor
\begin{equation}
R_{\mu\nu\rho\sigma} = \frac{1}{2}(h_{\mu\sigma, \nu\rho} + h_{\nu\rho,\mu\sigma} - h_{\nu\sigma, \mu\rho} - h_{\mu\rho, \nu\sigma}).
\end{equation}
And Ricci tensor
\begin{equation}
R_{\mu\nu} = \partial^\rho \partial_{(\mu} h_{\nu)\rho} - \frac{1}{2} \square h_{\mu\nu} - \frac{1}{2} \partial_\mu \partial_\nu h,
\end{equation}
where $h \coloneqq {h^\mu}_\mu$. It is also useful to define the `negative-traced' version of $h_{\mu\nu}$
\begin{equation}
\bar{h}_{\mu\nu} = h_{mu\nu} - \frac{1}{2}h\eta_{\mu\nu}.
\end{equation}
\end{noteEquation}

\begin{noteEquation}[Gauge Transformation of $h_{\mu\nu}$]
We generate `infinitesimal' diffeomorphisms using the flow of some vector field $X$ for a sufficiently small parameter $t$, which is equivalent to taking the flow with unit parameter of the vector field $\xi = tX$. The result is
\begin{equation}
h_{\mu\nu} \to h_{\mu\nu} + \xi_{\mu,\nu} + \xi_{\nu,\mu}.
\end{equation}
\end{noteEquation}

\begin{noteEquation}[Linearised Einstein Equation (Harmonic Gauge)]
\begin{equation}
\square \bar{h}_{\mu\nu} = -16 \pi T_{\mu\nu}
\end{equation}
\end{noteEquation}

\begin{noteEquation}[Gravitational Wave]
By solving the linearised vacuum Einstein equation with the gauge conditions $\partial^\mu \bar{h}_{\mu\nu} = 0$ (Lorentz), $\bar{h}_{0\mu} = 0$ (longitudinal gauge) and $\bar{h} = 0$ (trace-free), which can all be imposed concurrently, the most general form has just two degrees of freedom
\begin{equation}
\bar{h}_{\mu\nu} = h_{\mu\nu} = \begin{pmatrix}
0&0&0&0 \\
0 & H_+   &  H_\times & 0 \\
0 & H_\times & - H_+   & 0 \\
0&0&0&0
\end{pmatrix} e^{i k_\rho x^\rho}.
\end{equation}
\end{noteEquation}

\end{document}
