\documentclass[12pt]{article}

\usepackage{../noteStructure}

%\usepackage[
%	active,
%	generate=UltraBrief,,
%	extract-cmd= section,
%	extract-env={noteEquation}
%]{extract}
%\begin{extract}
%\usepackage{../../../mainStructure}
%\end{extract}

\title{General Relativity\\
\Large{Summary Notes}}
\author{Sam Crawford}
\date{\today}

\begin{document}
\maketitle

\tableofcontents

\pagebreak

\section{Equivalence Principles}

\begin{noteEquation}[Newton's Law of Gravitation]
	The differential form of Newtonian gravity is
		\begin{equation}
			\Delta \Phi = 4 \pi G \rho.
		\end{equation}
	The integral solution to this is
		\begin{equation}
			\phi(t,\mathbf{x}) = - G \int_{\mathbb{R}^3} \mathrm{d}^3y \, \frac{
				\rho(t,\mathbf{y})
			}{
				| \mathbf{x} - \mathbf{y}|.
			}
		\end{equation}
\end{noteEquation}

\section{Manifolds and Tensors}

\begin{prop}[Transformations of Vector and Covector Fields]\label{prop:trans1}
	Let $X$ be a vector field, and $\omega$ a covector field on $M$. Further, let $(x^\mu), (x'^\mu)$ be two sets of coordinates on $M$ with overlapping charts. If $X = X^\mu \partial_\mu = X'^\mu \partial'_\mu$ etc, then these coordinates are related to each other by
		\begin{subequations}\begin{align}
			X'^\mu &= \left( \frac{\partial x'^\mu }{\partial x^\nu} \right) X^\nu, \\
			\omega'_\mu &= \left( \frac{\partial x^\nu}{\partial x'^\mu} \right) \omega_\nu.
		\end{align}\end{subequations}
\end{prop}

\begin{definition}[Tensor]
	A \textbf{tensor of rank} $(r,s)$ on a vector field $V$ is a multilinear map
		\begin{equation}
			T: \underbrace{V^* \times \cdots \times V^*}_{r \text{ times}} \times \underbrace{V \times \cdots \times V}_{s \text{ times}} \to \mathbb{R}.
		\end{equation}
	Written in a basis $\{e_\mu\}$ with dual $\{ f^\mu \}$, such a tensor has components written
		\begin{equation}
			T = T^{\mu_1 \cdots \mu_r}_{\phantom{\mu_1 \cdots \mu_r} \nu_1 \cdots \nu_s} e_{\mu_1} \otimes \cdots \otimes e_{\mu_r} \otimes f^{\nu_1} \otimes \cdots \otimes f^{\nu_s}.
		\end{equation}
\end{definition}
\begin{remark}
	Remember a type $(r,s)$ vector `takes in' $r$ covectors and $s$ vectors in order to `put out' a scalar. If we instead input $p < r$ covectors and $q < s$ vectors, the result can be considered a $(r-p,s-q)$ tensor.
\end{remark}

\begin{definition}[Tensor Product]
	Given a rank $(p,q)$ tensor $S$ and a rank $(r,s)$ tensor $T$ on some vector space, their \textbf{tensor product}, a.k.a. \textbf{outer product}, is the rank $(p+r,q+s)$ tensor defined by
		\begin{align} \begin{split}
			(S \otimes T)(\omega_1, \ldots, \omega_p, \eta_1, \ldots, \eta_r, X_1, \ldots, X_q, Y_1, \ldots, Y_s) \\
			= S(\omega_1, \ldots, \omega_p, X_1, \ldots, X_q) T(\eta_1, \ldots, \eta_r, Y_1, \ldots, Y_s).
		\end{split} \end{align}
\end{definition}

\section{The Metric Tensor}

\begin{noteEquation}[Geodesic Lagrangian]\label{geoLagrange}
	To find geodesics on a Lorentzian manifold, we use a functional formula for the proper time, treating this as an action, the `Lagrangian' is
		\begin{equation}
			G(x(\lambda),\dot{x}(\lambda)) \coloneqq \sqrt{-g_{\mu\nu}(x) \dot{x}^\mu \dot{x}^\nu}.
		\end{equation}
	The proper time for a curve $x: [0,1] \hookrightarrow M$ is then
		\begin{equation}
			\tau[x] = \int_0^1 G(x(\lambda),\dot{x}(\lambda)) \mathrm{d}\lambda.
		\end{equation}
\end{noteEquation}
\begin{remark}
	The relevant derivatives to extremise the proper time are
		\begin{subequations} \begin{align}
			\frac{\partial G}{\partial \dot{x}^\mu} &= -\frac{1}{G} g_{\mu\nu} \dot{x}^\nu, \\
			\frac{\partial G}{\partial x^\mu} &= - \frac{1}{2G}g_{\nu \rho,\mu} \dot{x}^\nu \dot{x}^\rho.
		\end{align} \end{subequations}
\end{remark}

\begin{noteEquation}[Geodesic Equation]
	The Euler-Lagrange equations for \autoref{geoLagrange} reduce to the \textbf{geodesic equation}
		\begin{equation}\label{eq:geodesic}
			\frac{d^2 x^\mu }{d \tau^2} + \Gamma^\mu_{\nu\rho} \frac{d x^\nu}{d\tau} \frac{dx^\rho}{d\tau} = 0,
		\end{equation}
	where the \textbf{Christoffel symbols} are defined by
		\begin{equation}\label{eq:Christoffel}
			\Gamma^\mu_{\nu\rho} \coloneqq \tfrac{1}{2} g^{\mu\sigma} ( g_{\sigma \nu, \rho} + g_{\sigma \rho, \nu} - g_{\nu \rho, \sigma} ).
		\end{equation}
\end{noteEquation}
\begin{remark}
	One can obtain \eqref{eq:geodesic} more directly by varying the Lagrangian
		\begin{equation}
			L = -g_{\mu\nu} \frac{dx^\mu }{d\tau}\frac{dx^\nu}{d\tau}.
		\end{equation}
	Note further that the fact $L$ above has no explicit $\tau$ dependence, $\nicefrac{\partial L}{\partial \tau} = 0$ which, along with the realisation that $\nicefrac{dx^\mu}{d\tau}$ is a $4$-velocity, leads to the conclusion that $L \equiv -1$ along geodesics.
\end{remark}

\begin{example}[Schwarzschild Metric]
	One of the more famous solutions to Einstein's equations, the Schwarzschild metric can be written as
		\begin{equation}
			ds^2 = -f dt^2 + f^{-1} dr^2 + r^2 d\theta^2 + r^2 \sin^2\theta d \phi^2, \qquad f \coloneqq 1 - \frac{2M}{r}.
		\end{equation}
\end{example}

\section{Covariant Derivative}

\begin{definition}
	A \textbf{covariant derivative}, a.k.a. \textbf{connexion/connection} $\nabla$ on a manifold $M$ is a `variable' tensor field, taking in a vector field $X$ and a rank $(r,s)$ tensor field $T$ and producing a new rank $(r,s)$ tensor field, written $\nabla_X T$, subject to the following properties
		\begin{enumerate}
			\item $\nabla_{fX + gY} T = f \nabla_X T + g \nabla_Y T,$
			\item $\nabla_X(T + S) = \nabla_X T + \nabla_X S, $
			\item $\nabla_X(T \otimes S) = \left( \nabla_X T \right) \otimes S + T \otimes \left( \nabla_X S \right).$
			\item $\nabla_X f = X \circ f$
		\end{enumerate}
	I.e. the covariant derivative is $C^\infty(M)$ linear in the vector field argument, $\mathbb{R}$ linear in the tensor field argument, and satisfies the Leibniz rule for tensor products.
\end{definition}
\begin{remark}
	Sometimes we may wish to leave the vector field argument undefined, in which case we can consider $\nabla T$ as a type $(r,s+1)$ tensor field such that $\nabla T (X) = \nabla_X T$. In particular, for functions we have that $\nabla f = df$.
\end{remark}

\begin{noteEquation}[Tensor Coordinate Transformation]
	The generalisation of \autoref{prop:trans1} for an arbitrary $(r,s)$ tensor field is simply
		\begin{equation}
			T'^{\mu_1 \cdots \mu_r}_{\phantom{\mu_1 \cdots \mu_r} \nu_1 \cdots \nu_s} = \left( \frac{\partial x'^{\mu_1}}{\partial x^{\rho_1}} \right) \cdots \left( \frac{\partial x'^{\mu_r}}{\partial x^{\rho_r}} \right) \left( \frac{\partial x^{\sigma_1}}{\partial x'^{\nu_1}} \right) \cdots \left( \frac{\partial x^{\sigma_s}}{\partial x'^{\nu_s}} \right) T^{\rho_1 \cdots \rho_r}_{\phantom{\rho_1 \cdots \rho_r} \sigma_1 \cdots \sigma_s}
		\end{equation}
\end{noteEquation}

\begin{definition}
	Given a local basis $\{e_\mu \}$ of $\mathfrak{X}(M)$, a covariant derivative can be defined using \textbf{connection components} defined by
		\begin{equation}
			\nabla_\mu (e_\nu) \coloneqq \nabla_{e_\mu} (e_\nu) = \Gamma^\rho_{\nu\mu} e_\rho.
		\end{equation}
	The covariant derivative of a type $(1,1)$ tensor field can then be written as
		\begin{equation}
			T^{\mu}_{\nu ; \rho} \coloneqq (\nabla T)^\mu_{\nu \rho} = T^\mu_{\nu, \rho} + \Gamma^\mu_{\sigma \rho} T^\sigma_\nu - \Gamma^\sigma_{\nu \rho} T^\mu_\sigma.
		\end{equation}
	Transformations for arbitrary tensor fields look similar, but a lot messier.
\end{definition}
\begin{remark}
	For a scalar function we have $f_{; \mu} = f_{,\mu}$, a further covariant derivative is then given by
		\begin{equation}
			f_{;\mu\nu} = f_{,\mu\nu} - \Gamma ^\rho_{\mu\nu} f_{,\rho}.
		\end{equation}
\end{remark}

\begin{definition}[Torsion]
	The \textbf{torsion tensor} associated to a connexion $\nabla$ is a rank $(1,2)$ tensor $T$ defined by
		\begin{equation}
			T(X,Y) = \nabla_X Y - \nabla_Y X - [X,Y].
		\end{equation}
	If the torsion tensor vanishes everywhere, we say that the connexion is \textbf{torsion free}, in which case in \textit{any} basis $\Gamma^\rho_{\mu \nu} = \Gamma^\rho_{\nu\mu}$.
\end{definition}

\begin{theorem}[Fundamental Theorem of (Pseudo-)Riemannian Geometry]
	For any Riemannian manifold $(M,g)$, there exists a unique torsion-free connexion $\nabla$ such that $\nabla g \equiv 0$, called the \textbf{Levi-Civita connexion}. Locally, its connexion components are the same Christoffel symbols as defined in \eqref{eq:Christoffel}.
\end{theorem}

\begin{definition}[Affinely Parametrised Geodesic]
	An \textbf{affinely parametrised geodesic} on a manifold $M$ with connexion $\nabla$ is a curve with an associated vector field $X$ such that
		\begin{equation}
			\nabla_X X = 0.
		\end{equation}
\end{definition}
\begin{remark}
	This condition earns the `\textit{affine}' tag as reparametrising the curve $t \to t(u)$ results in the associated vector field $X \to Y = t' X$. Whilst this describes the same curve, if $\nabla_X X = 0$, then $\nabla_Y Y = (X \circ t') Y \neq 0$.
\end{remark}

\begin{theorem}
	Let $M$ be a manifold with connexion $\nabla$. Let $p \in M, X_p \in T_pM$. Then there exists a unique affinely parameterised geodesic through p with tangent vector $X|_p$ at $p$.
\end{theorem}
\begin{proof}
	Existence and uniqueness of solutions to ODEs. Specifically we want to find a curve $\gamma$ such that $\nabla_{\dot{\gamma}}\dot{\gamma} = 0$ subject to initial conditions $\gamma(0) = p$, $\dot{\gamma}(0) = X|_p$, which is a second order ODE.
\end{proof}

\begin{definition}[Exponential Map]
	Let $M$ be a manifold with connexion $\nabla$. For $p \in M$, the \textbf{exponential map} is a local diffeomorphism $\text{Exp}: T_pM \simto U \subset M$, defined such that if $\gamma$ is the geodesic through $p$ with $\dot{\gamma}_0 = X|_p$, then $\text{Exp}(X|_p) = \gamma(1)$.
\end{definition}
\begin{remark}
	It can be shown that, if $X|_p$ and $\gamma$ are defined as above, for $t \in [0,1]$
		\begin{equation}
			\text{Exp}(tX|p) = \gamma(t).
		\end{equation}
\end{remark}

\begin{definition}[Normal Coordinates]
	Given a manifold $M$ with connexion $\nabla$, and a basis $\{ e_\mu \}$ of $T_pM$, \textbf{normal coordinates} at $p$ are defined as the inverse of the map
		\begin{equation}
			(x^\mu) \mapsto \text{Exp}(x^\mu e_\mu).
		\end{equation}
	I.e. if $q = \text{Exp}(x^\mu e_\mu)$, then $x^\mu$ are the normal coordinates of $q$.
\end{definition}

\begin{lemma}
	For a manifold $M$ with connexion $\nabla$, in normal coordinates at $p \in M$, $\gamma^\mu_{(\nu\rho)} = 0$.
\end{lemma}
\begin{proof}
	We can express any geodesic from $p$ to $q = \text{Exp}(x^\mu e_\mu)$ in normal coordinates as
		\begin{equation}
			\gamma(t) = \text{Exp}(t x^\mu e_\mu),
		\end{equation}
	i.e. the coordinates of the geodesic are $(tx^\mu)$. Thus the geodesic equation becomes
		\begin{equation}
			\ddot{\gamma}^\mu + \Gamma^\mu_{\nu\rho} \dot{\gamma}^\nu \dot{\gamma}^\rho = 0 + \Gamma^\mu_{\nu\rho} x^\nu x^\rho = 0.
		\end{equation}
	As the Christoffel terms are being contracted with a symmetric term, we are free to symmetrise
		\begin{equation}
			\Gamma^\mu_{\nu\rho} x^\nu x^\rho = \Gamma^\mu_{(\nu\rho)} x^\nu x^\rho = 0.
		\end{equation}
\end{proof}

\begin{lemma}
	Let $M$ be a manifold with Levi-Civita connexion $\nabla$, then, in normal coordinates
		\begin{equation}
			g_{\mu\nu,\rho}(p) = 0.
		\end{equation}
	Further more, one can select a basis of $T_pM$ such that the metric at $p$ is $\text{Diag}(-, \cdots, +, \cdots)$, depending on the signature of the metric.
\end{lemma}
\begin{definition}
	For a Lorentzian manifold, a set of normal coordinates with respect to the Levi-Civita connexion such that $g_{\mu\nu}(p) = \eta_{\mu\nu}$ form a {\textbf{local inertial frame} at $p$}.
\end{definition}

\section{Physical Laws in Curved Spacetime}

\begin{definition}[Minimal Coupling]
	Given a set of equations of motion on a flat space time, the process of \textbf{minimal coupling} is defined to be the following
		\begin{ronumerate}
			\item Replace the Minkowski metric with a (generally) curved spacetime metric.
			\item Replace partial derivatives with covariant derivatives with respect to the Levi-Civita connexion
			\item Replace coordinate basis indices with abstract indices.
		\end{ronumerate}
\end{definition}

\begin{example}
	\begin{ronumerate}
		\item {[Klein-Gordon equation]} $\nabla^a \nabla_a \Phi - m^2 \Phi = 0.$
		\item {[Maxwell's Electromagnetism]} The field strength tensor is defined by $F_{ab} = \nabla_a A_b - \nabla_b A_a$, in a specific basis, this is related to the physical fields by $F_{0i} = - E_i$, $F_{ij} = \epsilon_{ijk} B_k$. The vacuum equations are then
			\begin{equation}
				\nabla^a F_{ab} = 0, \qquad \nabla_{[a} F_{bc]} = 0.
			\end{equation}
		The latter is a \textit{Bianchi identity}, and holds even in the presence of sources. The coupling of matter to $F$ is achieved through minimally coupling the Lorentz force law
			\begin{equation}
				u^b \nabla_b u^a = \tfrac{q}{m} {F^a}_b u^b.
			\end{equation}
	\end{ronumerate}
\end{example}

\begin{example}[Stress-Energy Tensors]
	Each of the above theories has a corresponding stress-energy tensor which is symmetric and \textit{conserved}, i.e. $\nabla^a T_{ab} = 0$.
		\begin{ronumerate}
			\item {[Klein-Gordon Field]} DO
			\item {[Maxwell's Electromagnetism]} $T_ab = \frac{1}{4 \pi}\left( F_{ac} {F_b}^c- \frac{1}{4}F^{cd}F_{cd} g_{ab} \right)$
			\item {[Perfect fluid]} $T_{ab} = (\rho + p) u_a u_b + pg_{ab}$
		\end{ronumerate}
\end{example}

\section{Curvature}

\begin{definition}[Parallel Transport]
	Given a manifold $M$ with connexion $\nabla$ and a vector field $X$, a tensor field $T$ is said to have undergone \textbf{parallel transport} with respect to $X$ if $\nabla_X T = 0$.
\end{definition}
\begin{remark}
	Given the value of $T$ at $p \in M$, the parallel transport condition uniquely determines the value of $T$ for all points along the integral curve of $X$ through $p$.
\end{remark}

\begin{definition}[Riemann Curvature]
	Given a manifold $M$ with connexion $\nabla$, the \textbf{Riemann curvature tensor} of the connexion is a type $(1,3)$ tensor defined by
		\begin{equation}
			R(X,Y)Z = \nabla_X \nabla_Y Z - \nabla_Y \nabla_X Z - \nabla_{[X,Y]} Z.
		\end{equation}
\end{definition}
\begin{remark}
	The fact that this is indeed a tensor is not trivial. Basically we must prove that $R(fX + X', Y) = fR(X,Y) + R(X',Y)$ and similar for $Z$ ( $C^\infty(M)$ linearity in $Y$ is given by the inherent skew-symmetry of $R$ in $X$ and $Y$).
\end{remark}

\begin{noteEquation}
	By computing $R(e_\rho, e_\sigma) e_\nu$, we can obtain the basis-dependent form of the Riemann tensor
		\begin{equation}
			{R^\mu}_{\nu \rho \sigma} = \partial_\rho \Gamma^\mu_{\nu\sigma}
 - \partial_\sigma \Gamma^\mu_{\nu\rho} + \Gamma^\tau_{\nu\sigma}\Gamma^\mu_{\tau\rho} - \Gamma^\tau_{\nu\rho}\Gamma^\mu_{\tau\sigma}.
 		\end{equation}
\end{noteEquation}

\begin{definition}[Ricci Curvature Tensor]
	The \textbf{Ricci curvature tensor} is a contraction of the Riemann tensor, defined by
		\begin{equation}
			R_{ab} \coloneqq {R^c}_{acb}.
		\end{equation}
\end{definition}

\begin{noteEquation}[Ricci Identity]
	\begin{equation}
		\nabla_c \nabla_d Z^a - \nabla_d \nabla_c Z^a = {R^a}_{bcd} Z^b.
	\end{equation}
\end{noteEquation}
(Proof: contract each side with $X^c, Y^d$)

\begin{noteEquation}[Symmetries of the Riemann Tensor]
		\begin{equation}
			{R^a}_{b(cd)} = 0.
		\end{equation}
	For a torsion-free connexion:
		\begin{equation}
			{R^a}_{[bcd]} = 0.
		\end{equation}
	\textit{Bianchi identity:} (also for a torsion-free connexion)
		\begin{equation}
			{R^a}_{b[cd;e]} = 0.
		\end{equation}
	If $\nabla$ is the Levi-Civita connexion for some metric $g_{ab}$, then
		\begin{equation}
			R_{abcd} = R_{cdab}, \qquad R_{(ab)cd} = 0.
		\end{equation}
\end{noteEquation}

\begin{noteEquation}[Geodesic Deviation]
	Let $\nabla$ be a torsion-free connection, and let $T,S$ be vector fields such that $\nabla_T T = 0$, and $[T,S] = 0$. Then
		\begin{equation}
			\nabla_T \nabla_T S = R(T,S) T.
		\end{equation}
\end{noteEquation}

\begin{definition}[Einstein Tensor]
	The \textbf{Einstein tensor} is a symmetric tensor of type $(0,2)$ defined by
		\begin{equation}
			G_{ab} \coloneqq R_{ab} - \frac{1}{2}Rg_{ab}
		\end{equation}
\end{definition}

\begin{noteEquation}[Contracted Bianchi Identity]
	\begin{align}
		\nabla^aG_{ab} = 0, \quad \Leftrightarrow \quad \nabla^a R_{ab} - \frac{1}{2} \nabla_b R = 0.
	\end{align}
\end{noteEquation}

\begin{noteEquation}[Einstein Equation]
	\begin{equation}
		G_{ab} = 8\pi G T_{ab}.
	\end{equation}
	If a vacuum, this reduces to
	\begin{equation}
		R_{ab} = 0.
	\end{equation}
	If we wish to include the cosmological constant, we then have
	\begin{equation}
		G_{ab} + \Lambda g_{ab} = 8 \pi G T_{ab}.
	\end{equation}
\end{noteEquation}

\section{Diffeomorphisms and the Lie Derivative}


\begin{definition}[Pullback (function)]
	Let $\phi: M \to N$ be a smooth map between manifolds (not necessarily a diffeomorphism), the \textbf{pullback} of a function $f \in C^\infty(N)$ is the $C^\infty(M)$ function defined by
		\begin{equation}
			\phi^* f \coloneqq f \circ \phi.
		\end{equation}
\end{definition}

\begin{definition}[Pushforward]
	Let $\phi: M \to N$ be a smooth map between manifolds. Then the \textbf{pushforward} of $\phi$ is a linear map $\phi_*: T_pM \to T_pN$ defined by
		\begin{equation}
			(\phi_* v) f = v (\phi^* f)
		\end{equation}
	where $f \in C^\infty(N)$ is an arbitrary function.
\end{definition}
\begin{remark}
	An alternate definition is that the pushforward $\phi_*X$ of $X$ is the vector field on $N$ such that if $\rho_t$ is the flow of $X$ on $M$, then $\phi \circ \rho_t$ is the flow of $\phi_* X$.
\end{remark}

\begin{definition}[Pullback (differential form)]
	Using these, we can define the \textbf{pullback} of a $p$-form $\alpha \in \Omega^p(N)$ to be
		\begin{equation}
			(\phi^* \alpha) (X_1, \cdots, X_p) = \alpha( \phi_* X_1, \cdots, \phi_* X_n),
		\end{equation}
	for arbitrary $X_1, \cdots X_n \in \mathfrak{X}(M)$.
\end{definition}
\begin{remark}
	We can infact define pullbacks for \textit{any} type $(0,s)$ tensors, not just those which are antisymmetric, and likewise we can define pushforwards for type $(r,0)$ tensors using similar `antidistributivity' relations. But these are less useful.
\end{remark}

\begin{lemma}
	The exterior derivative commutes with pullbacks, i.e.
		\begin{equation}
			\phi^* (d\alpha) = d(\phi^* \alpha).
		\end{equation}
\end{lemma}
(Only expected to prove this for $0$-forms.)
\begin{remark}
	The pullback also commutes with contractions.
\end{remark}

\begin{definition}[Diffeomorphism]
	A map $ \phi: M \to N$ is a \textbf{diffeomorphism} if it is bijective, smooth, and has a smooth inverse
\end{definition}

\begin{definition}[Pullback/Pushforward (of a Diffeomorphism)]
	Using a diffeomorphism, we can pullback vector fields, and pushforward differential forms using the inverse, i.e.
		\begin{equation}
			\phi_* \alpha \coloneqq \left(\phi^{-1}\right)^* \alpha, \qquad
			\phi^* X \coloneqq \left( \phi^{-1} \right)_* X.
		\end{equation}
\end{definition}

\begin{noteEquation}[Coordinate Based Pullback/Pushforward]
	Let $\phi: M \to N$ be a diffeomorphism, and let $(x^\mu)$ be a set of coordinates on $M$, and $(y^\mu)$ a set of coordinates on $N$. Then
	\begin{align}
		\left( \phi_* \frac{\partial}{\partial x^\mu} \right) 
		= \left( \frac{\partial y^\nu}{\partial x^ \mu} \right) \frac{\partial }{\partial y^\nu}, 
			\qquad
		\left( \phi^* dy^\mu \right) = \left( \frac{\partial y^\mu }{\partial x^\nu} \right) dx^\nu.
	\end{align}
	The inverse operations, allowed by the fact that $\phi$ is a diffeomorphism, are then
	\begin{align}
		\left( \phi^* \frac{\partial}{\partial y^\mu} \right) 
		= \left( \frac{\partial x^\nu}{\partial y^\mu} \right) \frac{\partial}{\partial x^\nu},
			\qquad
		\left( \phi_* dx^\mu \right) 
		= \left(\frac{\partial x^\mu }{\partial y^\nu} \right) dy^\nu.
	\end{align}
	Thus, for a tensor $S$ of type $(r,s)$ on $M$, and $T$ on $N$, we have
	\begin{align}
		{\left( \phi_* S \right)^{\mu_1, \cdots, \mu_r}}_{\nu_1, \cdots \nu_s} = \left( \frac{\partial y^{\mu_1}}{\partial x^{\alpha_1}}\right) \cdots \left( \frac{\partial y^{\mu_r}}{\partial x^{\alpha_r}}\right)
		\left( \frac{\partial x^{\beta_1}}{\partial y^{\nu_1} }\right) \cdots
		\left( \frac{\partial x^{\beta_s}}{\partial y^{\nu_s} }\right)
		{S^{\alpha_1 \cdots \alpha_r}}_{\beta_1 \cdots \beta_s}.
	\end{align}
\end{noteEquation}
\begin{remark}
	Essentially, this is just the chain rule. In fact, if we have charts for $U \subset M$ and $\phi(U)$, then we can think of a diffeomorphism as essentially a change of chart. 
\end{remark}

\begin{definition}[Pushforward (of a Connexion)]
	Let $\phi: M \to N$ be a diffeomorphism. Then if we have a connexion $\nabla$ on $M$, we can define its \textbf{pushforward} $\tilde{\nabla}$ on $N$ by
		\begin{equation}
			\tilde{\nabla}_X T = \phi_* \left[ \nabla_{\phi^* X} ( \phi^* T) \right].
		\end{equation}
\end{definition}

\begin{lemma}
	Let $\phi: M \simto N$, and let $\tilde{\nabla}$ be the pushforward of $\nabla$. Then
		\begin{ronumerate}
			\item The Riemann tensor of $\tilde{\nabla}$ is the pushforward of the Riemann tensor of $\nabla$.
			\item If $\nabla$ is the Levi-Civita connexion of a metric $g$, then $\tilde{\nabla}$ is the Levi-Civita connexion of $\phi_*g$.
		\end{ronumerate}
\end{lemma}
\begin{remark}
	From these results (and other similar results), we conclude that diffeomorphisms represent a \textit{gauge freedom} in our description of GR.
\end{remark}

\begin{definition}[Symmetry Transformation/Isometry]
	A diffeomorphism $\phi: M \to M$ is a \textbf{symmetry transformation} of a tensor field $T$ if $\phi_* T \equiv T$. A symmetry transformation of the metric tensor is called an \textbf{isometry}.
\end{definition}

\begin{remark}
	A vector field $X \in \mathfrak{X}(M)$ generates a family of diffeomorphisms $\rho_t : M \simto M$ satisfying $\rho_t \circ \rho_s = \rho_{t+s}$, and $\rho_0 \equiv \text{Id}$. Thus $\rho_t^{-1} = \rho_{-t}$ and $\rho_t^* = (\rho_{-t})_*$.
\end{remark}

\begin{definition}
	The \textbf{Lie derivative} of a tensor field $T$ with respect to a vector field $X$ is defined using the flow $\rho_t$ of $X$ as
		\begin{equation}
			\mathcal{L}_X T = \frac{d}{dt}\left[ \rho_t^* T \right]|_{t=0}.
		\end{equation}
	Alternatively, we can define this more explicitly pointwise as
		\begin{equation}
			\mathcal{L}_X T|_p = \lim_{t \to 0} \frac{ 
				{\rho_t}^* T|_{\rho_t(p)} - T|_p
			}{
				t
			}.
		\end{equation}
\end{definition}

\begin{lemma}
	Given a vector field $X$ on a manifold $M$, and a set of coordinates $(t,x^i)$ adapted to $X$ such that $X(t) = 1, \, X(x^i) = 0 \Leftrightarrow X = \partial_t$, we can define the Lie derivative in this basis as
		\begin{equation}
			{\left( \mathcal{L}_X T \right)^{\mu_1, \cdots \mu_r}}_{\nu_1, \cdots \nu_s} = \frac{\partial}{\partial t} {T^{\mu_1 \cdots \mu_r}}_{\nu_1 \cdots \nu_s}.
		\end{equation}
\end{lemma}

\begin{remark}
	The Lie derivative also 
		\begin{itemize}
			\item Obeys the Leibniz rule: $\mathcal{L}_X (S \otimes T) = (\mathcal{L}_X S) \otimes T + S \otimes (\mathcal{L}_X T)$.
			\item Interacts with the contraction operator as: $[\iota_X, \mathcal{L}_Y] = \iota_{[X,Y]}$.
			\item Commutes with `internal' contraction: $\mathcal{L}_X ({T^{\cdots a \cdots}}_{\cdots a \cdots} ) = {(\mathcal{L}_X T)^{\cdots a \cdots}}_{\cdots a \cdots}.$
			\item Coincides with the action of $X$ on $0$-forms: $\mathcal{L}_X f = X(f)$.
			\item Coincides with the Lie bracket on vector fields: $\mathcal{L}_X Y = [X,Y]$
		\end{itemize}
\end{remark}

\begin{example}[Lie Derivative vs Covariant Derivative ($1$-forms)]
	The Lie derivative and covariant derivative agree on functions, we can use this to compare their action on generic tensors. For example, consider the action of the Lie derivative on a covariant vector field
		\begin{equation}
			\mathcal{L}_X (\iota_Y \omega) = X \left(\omega(Y) \right) = \iota_Y(\mathcal{L}_X \omega) = (\mathcal{L}_X \omega)(Y)
		\end{equation}
	[TO FINISH: compute in normal coords of Levi-Civita]
\end{example}

\begin{noteEquation}[Killing's Equation]
	A vector field $X$ is a \textbf{Killing vector field} if its flows are isometries of $g$, equivalently $\mathcal{L}_X g = 0$. This can be show to be equivalent to \textit{Killing's equation}
		\begin{equation}
			\nabla_a X_b + \nabla_b X_a = 0.
		\end{equation}
	Or, equivalently
\end{noteEquation}

\begin{lemma}
	If $X$ is a Killing vector field, and $\gamma$ is an affinely parametrised geodesic, then $g(X,\dot{\gamma})$ is constant along $\gamma$
\end{lemma}
\begin{proof}
	$\dot{\gamma}\left( g(X,\dot{\gamma} ) \right) = g( \nabla_{\dot{\gamma}} X, \dot{\gamma} ) = $
\end{proof}

\section{Linearised Theory}

\begin{noteEquation}[Linearly Perturbed Metric]
	Assuming that space is `almost flat' we can find a chart such that
		\begin{equation}
			g_{\mu\nu} = \eta_{\mu\nu} + h_{\mu\nu},
		\end{equation}
	such that $h_{\mu\nu} \sim \mathcal{O}(\epsilon)$ and $\epsilon^2 \approx 0$. The inverse metric is then
		\begin{equation}
		g^{\mu \nu} = \eta^{\mu\nu} + h^{\mu\nu},
		\end{equation}
	where $h^{\mu\nu} = \eta^{\mu\rho} \eta^{\nu\sigma} h_{\rho \sigma} + \mathcal{O}(\epsilon^2)$.
\end{noteEquation}

\begin{noteEquation}[Linearised Levi-Civita]
	Given the above linearisation, the Christoffel symbols of the Levi-Civita connexion can be written
		\begin{equation}
			\Gamma^\mu_{\nu\rho} = \frac{1}{2}
			\eta^{\mu\sigma} (h_{\sigma \nu,\rho} + h_{\sigma \rho, \nu} - h_{\nu\rho, \sigma} ).
		\end{equation}
	This leads to the linearised Riemann tensor
		\begin{equation}
			R_{\mu\nu\rho\sigma} = \frac{1}{2}(h_{\mu\sigma, \nu\rho} + h_{\nu\rho,\mu\sigma} - h_{\nu\sigma, \mu\rho} - h_{\mu\rho, \nu\sigma}).
		\end{equation}
	And Ricci tensor
		\begin{equation}
			R_{\mu\nu} = \partial^\rho \partial_{(\mu} h_{\nu)\rho} - \frac{1}{2} \square h_{\mu\nu} - \frac{1}{2} \partial_\mu \partial_\nu h,
		\end{equation}
	where $h \coloneqq {h^\mu}_\mu$. It is also useful to define the `negative-traced' version of $h_{\mu\nu}$
		\begin{equation}
			\bar{h}_{\mu\nu} = h_{mu\nu} - \frac{1}{2}h\eta_{\mu\nu}.
		\end{equation}
\end{noteEquation}

\begin{noteEquation}[Gauge Transformation of $h_{\mu\nu}$]
	We generate `infinitesimal' diffeomorphisms using the flow of some vector field $X$ for a sufficiently small parameter $t$, which is equivalent to taking the flow with unit parameter of the vector field $\xi = tX$. The result is
		\begin{equation}
			h_{\mu\nu} \to h_{\mu\nu} + \xi_{\mu,\nu} + \xi_{\nu,\mu}.
		\end{equation}
\end{noteEquation}

\begin{noteEquation}[Linearised Einstein Equation (Harmonic Gauge)]
	\begin{equation}
		\square \bar{h}_{\mu\nu} = -16 \pi T_{\mu\nu}
	\end{equation}
\end{noteEquation}

\begin{noteEquation}[Gravitational Wave]
	By seeking solutions of the linearised vacuum Einstein equation with the gauge conditions $\partial^\mu \bar{h}_{\mu\nu} = 0$ (Lorentz), $\bar{h}_{0\mu} = 0$ (longitudinal gauge) and $\bar{h} = 0$ (trace-free), which can all be imposed concurrently. For a plane wave solution with wave-vector $k_\rho = (\omega, 0, 0, \omega)$ the most general form has just two degrees of freedom
		\begin{equation}
			\bar{h}_{\mu\nu} = h_{\mu\nu} = \begin{pmatrix}
				0&0&0&0 \\ 
				0 & H_+ 	 & 	H_\times & 0 \\ 
				0 & H_\times & - H_+ 	 & 0 \\ 
				0&0&0&0
			\end{pmatrix} e^{i k_\rho x^\rho}.
		\end{equation}
\end{noteEquation}

\begin{noteEquation}[Linearised Geodesic Deviation]
	For a linear metric perturbation $h_{\mu\nu}$, and a parallelly transported orthonormal basis $\{(e_\alpha)^a\}$, we can approximate the geodesic deviation vector as
		\begin{equation}
			\frac{d^2 S_\alpha}{d \tau^2} \approx R_{\mu 0 0 \nu} e^\mu_\alpha e^\nu_\beta S^\beta \approx \tfrac{1}{2} \frac{\partial^2 h_{\mu\nu}}{\partial t^2} e^\mu_\alpha e^\nu_\beta S^\beta.
		\end{equation}
\end{noteEquation}

\end{document}